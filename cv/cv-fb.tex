\documentclass[10pt]{article}

\usepackage{geometry}
\geometry{letterpaper, margin=0.75in}
\setlength{\parindent}{0em}
\setlength{\parskip}{1em}
\renewcommand{\baselinestretch}{1.2}
\usepackage{fontawesome}
\usepackage[colorlinks=true, linkcolor=green, urlcolor=blue]{hyperref}
\usepackage{tabularx}

\begin{document}

\begin{center}
    \textbf{\LARGE Jiacheng (Gary) Liu}
    \hrule
    \faEnvelope \, liujch1998@gmail.com \quad
    \faEnvelope \, jl25@illinois.edu \quad
    \faLinkedin \, liujch1998 \quad
    \faGithub \, liujch1998 \quad
    \faGithubAlt \, liujch1998.github.io
\end{center}

\vspace{1em}
{\LARGE EDUCATION}

\hangindent=1em
\textbf{UNIVERSITY OF ILLINOIS AT URBANA-CHAMPAIGN (UIUC)} \hfill Urbana, IL, USA \\
B.S. in Computer Science; Minor in Physics; GPA 3.97/4.0 \hfill 2016.08 - 2019.12 \\
James Scholar; Dean's List: 2016 Fall, 2017 Spring, 2017 Fall, 2018 Spring, 2018 Fall, 2019 Spring \\
Teaching: CS 498-VR Virtual Reality, Course Assistant, 2018.01-2018.05

\vspace{1em}
{\LARGE INDUSTRY EXPERIENCE}

\hangindent=1em
\textbf{Facebook} \hfill Software Engineer \hfill 2020.02 - Present \\
Design core, backend software components, and code using primarily C/C++, Java, and PHP \\
Interface with other teams to incorporate their innovations and vice versa \\
Conduct Design and code reviews, and analyze and improve efficiency, scalability, and stability of various system resources

\hangindent=1em
\textbf{Oculus, Facebook} \hfill Software Engineering Intern \hfill 2019.05 - 2019.08 \\
Optimized graphics pipeline in Oculus Quest \\
Built internal tools supporting better evaluation of graphics quality \\
Focused on visual fidelity and latency for better immersion

\hangindent=1em
\textbf{Exegy} \hfill Software Engineering Intern \hfill 2017.05 - 2017.08 \\
Conducted profile-guided optimization (PGO) on the core software of Exegy \\
Achieved significant reduction on latency of market data feeds normalization and transformation \\
Integrated automated PGO workflow into build system \\
Gained proficiency in OOP and template programming in C++, and experience with GCC and Linux

\vspace{1em}
{\LARGE PROJECTS}

\hangindent=1em
\textbf{Machine Learning Lab - a collection of experimental ideas and preliminary implementations} \\
\textbf{RhymeNet:} substituting rhymes in Chinese Song poems (Python \& PyTorch) $|$ \href{https://github.com/liujch1998/Lab/tree/master/ml-songci}{Code} \hfill 2018.08 \\
\textbf{DFTnet:} efficiently training large neural networks (Python \& PyTorch) $|$ \href{https://github.com/liujch1998/Lab/blob/master/ml-dft-nn/report/report.pdf}{Report} $|$ \href{https://github.com/liujch1998/Lab/tree/master/ml-dft-nn}{Code} \hfill 2018.05

\hangindent=1em
\textbf{BiReality - a virtual reality world infrastructure} \hfill 2015.09 - Present \\
\textit{A universal virtual space platform that simulates, complements and extends reality. Present projections of real-world landscapes, architecture and objects with corresponding functionality, while allowing pure creations. Individual users can enjoy living in residence, conduct activities, and lead another life. Public services (libraries) and commercial activities (shops, sport events, concerts) can be performed in virtual space. Aims at offering spatial accessibility and integrating VR platforms.} \\
Designed and developed infrastructure for universal virtual world platform \\
Implemented client with C\#, Unity; console with Vue.js, JavaScript, Java \\
Constructed a demo world with common utility venues (e.g. libraries, galleries, furniture stores)

\hangindent=1em
\textbf{Experience Illinois - a VR experience of UIUC campus (CS 498 team project)} $|$ \href{https://github.com/liujch1998/Illinois-Experience}{Code} \hfill 2017.10 - 2017.12 \\
\textit{A virtual tour of UIUC featuring panoramic view of various campus locations, group tours, interactive activities. } \\
Developed multiuser game Frisbee-on-the-quad; applied depth correction (Oculus, Unity, C\#, Photon networking) \\
Developed book reading experience in library scene; researched on high-quality text displaying in VR \\
Endorsed by Prof. Steve LaValle and Anna Yershova

\vspace{1em}
{\LARGE RESEARCH EXPERIENCE}

\hangindent=1em
\textbf{Phrase Grounding (with Prof. Julia Hockenmaier, Computer Science, UIUC)} \hfill 2018.06 - 2019.12 \\
Undergraduate Research Assistant $|$ \href{https://arxiv.org/abs/1909.00301}{Paper} $|$ \href{https://github.com/liujch1998/SoftLabelCCRF}{Code} \\
Approached the phrase grounding problem as a sequence labeling task \\
Extended standard CRFs to Soft-Label CRFs that adapt to the task by solving gold label multiplicity \\
Developed mathematical formulation and learning algorithm for Soft-Label Chain CRFs \\
Applied Soft-Label Chain CRFs to phrase grounding and improved state-of-the-art on Flickr30k Entities

\hangindent=1em
\textbf{Depth Correction in VR (with Prof. Anna Yershova, Computer Science, UIUC)} \hfill 2017.10 - 2018.05 \\
Independent Study $|$ \href{https://github.com/liujch1998/Lab/blob/master/vr-360-depth-correction/report/report.pdf}{Report} $|$ \href{https://github.com/liujch1998/Lab/tree/master/vr-360-depth-correction}{Code} \\
Studied the problem that virtual objects are not perceived as correctly located in 360 video environment \\
Derived 3D geometric transformation that provides corrected monocular depth cue in consensus with background

\hangindent=1em
\textbf{Adiabatic Quantum Computing (with Prof. Bryan Clark, Physics, UIUC)} \hfill 2017.08 - 2017.12 \\
Independent Study $|$ \href{https://github.com/liujch1998/AQC/blob/master/report/report.pdf}{Report} $|$ \href{https://github.com/liujch1998/AQC}{Code} \\
Analyzed the asymptotic time complexity of adiabatic quantum algorithms on selected NP-Complete problems (e.g. max clique, max vertex independent set, min vertex cover set) \\
Learned basics of adiabatic quantum computing (AQC); implemented numerical simulations of AQC and projector Monte Carlo on classical architecture

\vspace{1em}
{\LARGE PUBLICATIONS}

\hangindent=1em
\textbf{Jiacheng Liu} and Julia Hockenmaier. 2019. \emph{Phrase Grounding by Soft-Label Chain Conditional Random Field.} In Proceedings of the 2019 Conference on Empirical Methods in Natural Language Processing and 9th International Joint Conference on Natural Language Processing (EMNLP-IJCNLP 2019). \href{https://www.aclweb.org/anthology/D19-1515}{https://www.aclweb.org/anthology/D19-1515} (Long Paper, Oral)

\hangindent=1em
Zihan Wang, Jingbo Shang, Liyuan Liu, Lihao Lu, \textbf{Jiacheng Liu} and Jiawei Han. 2019. \emph{CrossWeigh: Training Named Entity Tagger from Imperfect Annotations.} In Proceedings of the 2019 Conference on Empirical Methods in Natural Language Processing and 9th International Joint Conference on Natural Language Processing (EMNLP-IJCNLP 2019). \href{https://www.aclweb.org/anthology/D19-1519}{https://www.aclweb.org/anthology/D19-1519} (Long Paper, Oral)

\vspace{1em}
{\LARGE AWARDS}

\begin{tabularx}{\textwidth}{X l r}
\textbf{2020 CRA Outstanding Undergraduate Researcher Award} & Honorable Mention & 2019.12 \\
\textbf{Correlation One Terminal Live: UIUC} & Team 1st Place & 2019.09 \\
\textbf{John R. Pasta Outstanding Undergraduate Award} & & 2019.04 \\
\textbf{ACM-ICPC World Finals} & Team 62nd Place & 2019.04 \\
\textbf{ACM-ICPC Neural Network Challenge} & Team 2nd Place & 2019.04 \\
\textbf{ACM-ICPC Mid-Central USA Regional Programming Contest} & Team 1st Place & 2018.11 \\
\textbf{UI Undergraduate Math Contest} & 1st Place & 2018.02 \\
\textbf{ACM-ICPC Mid-Central USA Regional Programming Contest} & Team 3rd Place & 2017.11 \\
\textbf{ACM-ICPC Mid-Central USA Regional Programming Contest} & Team 4th Place & 2016.11
\end{tabularx}

\end{document}
